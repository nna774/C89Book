\documentclass[twocolumn,12pt,b5book]{extarticle}

\usepackage{xltxtra}
\setmainfont[Ligatures=TeX]{IPAPMincho}
\setsansfont{IPAPGothic}
\setmonofont{IPAGothic}
\XeTeXlinebreaklocale "ja"

\usepackage{hyperref}
\usepackage{listings}
\lstset{%
 backgroundcolor={\color[gray]{.85}},%
 basicstyle={\small},%
 frame={tb},
 breaklines=true,
 columns=[l]{fullflexible},%
 lineskip=-0.5ex%
}
\usepackage{verbatim}
%\usepackage{biblatex}
\usepackage[style=authoryear,backend=bibtex]{biblatex}
\nocite{*}
\addbibresource{ref}
\usepackage{graphicx}
\usepackage{ulem}
\usepackage{myjapanese}

%\usepackage{endnotes}
%\let\footnote=\endnote
\newcommand{\vs}{\vspace{\baselineskip}}

\begin{document}
%\pagestyle{fancy}

\cleardoublepage
\tableofcontents

\cleardoublepage
% \mainmatter

%\renewcommand{\baselinestretch}{0.9}


\part{まえがき}
\section{はじめに}
はじめまして、もしくはいつもありがとうございます、NoNameA 774でございます。
今回はPietに関する本というなんともニッチな本をお読みいただき、ありがとうございます。

今回のC89で新しく、サークル「いっと☆わーくす!」を立ち上げ、コミケに参加させて頂きました。
簡単にここまでのあらすじを書いておくと、前回C89で友人がサークル申し込みを初めてやってみるというので、
「通ったら何か置かせて」のようなことを言っていたらほんとに通ってしまったので、
前回Pietに関するペーパーを置かせてもらいました
\footnote{その時配布したものは\url{https://nna774.net/piet/}にリンクが有ります。
ペーパーだからすぐに印刷できると思ってギリギリまで書かなかったのですごく雑な出来だったので反省しています……。
出ないよりはマシということで(この本も)、とりあえずpublishしている感じです。
現在時刻は2015/11/12 5:15ごろ。レポジトリ作ったのはもうちょっと前ですが、中身を書き始めたのはさっきです。
今回のはもうちょっとマシに仕上げたいですね……。}。
友人のサークルは前回てーきゅうに関する本だったので、二日目のアース・スターゾーンに配置されました。
そんな中で配布したPietのペーパーでしたが、何枚か知り合い以外にも持って行って下さった人がいました。
ありがとうございました。(残りのペーパーは次の日三日目、KMCブースで来た人に押し付けてもらいました。
可能性としてはそっちで手にとっていただいた人のほうが多いのでしょうか? そもそも30枚程度しか刷っていませんが……)
まぁそのような感じで、C88は参加していたのですが、
一度ぐらいはサークル参加をしてみたいなと以前から思っていたので
(一度きりになるのかまた自分のサークルで出たいと思うのかはこのC89が終わらないとわかりませんが)、
申し込んでみたところめでたく当選したので、このPietに関する本を書いています。
夏コミ周辺で結構お金を使ってしまって\footnote{このへんの話は同じくこのC89三日目で出るKMCの部誌に書く予定です。遠野に旅行とかしてました。}
今回あまりお金が無かったので、
抽選に落ちたら行かないかなー とすら思っていたのですが、
今回当選したのでがんばります。

サークルの名前「いっと☆わーくす!」ですが、私のWebページのタイトルとして使っていた文字列で
\footnote{元ネタはApacheのデフォルトページのIt works!です。}、一人サークルです。
KMC\footnote{京大マイコンクラブ。私の所属しているサークルで京大でコンピュータ関連のなんでもをしているサークルです。}
の部誌として出さない意義はあるのか とかいう話もありますが、とりあえず一度自分のサークルで参加というものをしてみたかった感じです。


TODO: このへん後でまとめてほしい。

\cleardoublepage

\part{Piet}

\section{1章}

あ\footnote{い}
a\cite{wppiet}

\printbibliography 

\vs
\hrule
%\theendnotes

\null\vfill
\section*{奥付}
\hrule\vskip.5mm\hrule\vskip3mm
\begin{tabular}{ll}
2015/12/31 & 初版発行\\
hash: & \href{}{hoge}(の次)\\
著作・発行 & NoNameA 774 (nonamea774@nnn77) \\
メールアドレス & \href{mailto:nonamea774@gmail.com}{\nolinkurl{nonamea774@gmail.com}}\\
Web & \url{https://nna774.net/}\\
Twitter & @nonamea774\\
GPG Key & 0x0C3E3AB2\\
fingerprint & 674A 287A 21D2 2431 AD8F \\
 & D328 AEF3 C3C7 0C3E 3AB2
\end{tabular}
\vskip3mm\hrule\vskip3mm

This article is licensed unser GFDL 1.3 or any later versions. And/or CC BY-SA 4.0 International.
You can get a machine-readable Transparent copy from \url{https://github.com/nna774/C89Book}
\end{document}
